\section{Conclusion}
To conclude, we have seen that the KPIs set as a target by the University itself are not that well defined and certainly not achieved in most cases. Elaborating on this, the results show that the University at least tries some form of implementing the KPIs in the timetable algorithms. However, we suspect that there are many improvements possible to boost the compliance rates of the KPIs.

Moreover, with more data and specifically more accurate and consistent data, it becomes possible to do a more in-depth analysis on different aspects. For example, it could be interesting to see how the different locations for a particular student are spread out around the campus. With the current data set, this is nearly impossible but when more data is available (possibly a dataset with distances between different locations) this becomes more doable.