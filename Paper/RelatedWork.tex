\section{Related Work}
The research of Burke and Petrovic \cite{recentResearchDirectionsInAutomatedTimetabling} introduces some approaches to timetabling problems that were developed or are under development at that time in the Automated Scheduling, Optimisation and Planning Research Group (ASAP) at the University of Nottingham. Their aim was basically the same as in our research since they specifically concentrated upon analysing and improving university timetabling as well. The paper suggests a number of approaches and comprises three parts. Firstly, recent heuristic and evolutionary timetabling algorithms are discussed. In particular, two evolutionary algorithm developments are described: a method for decomposing large real-world timetabling problems and a method for heuristic initialisation of the population. Secondly, an approach that considers timetabling problems as multi-criteria decision problems is presented. Thirdly, it discusses a case-based reasoning approach that employs previous experience to solve new timetabling problems which basically is similar to our research where we will be analysing past schedules to give indications for improving future schedules.

Also Burke et al. performed another study in 2004 \cite{burke2004analysing} specifically focussing on examination timetabling as a subset of the entire timetabling problem. In their paper they carry out an investigation of some of the major features of exam timetabling problems with a view to developing
a similarity measure. This similarity measure will be used within a case-based reasoning (CBR) system to match a new problem with one from a case-base of previously solved problems. The case base will also store the heuristic or meta-heuristic technique(s) applied most successfully to each problem stored. The technique(s) stored with the matched case will be retrieved and applied to the new case. The CBR assumption in their system is that similar problems can be solved equally well by the same technique.

Wang's paper \textit{Using genetic algorithm methods to solve course scheduling problems} \cite{usingGeneticAlgorithmMethodsToSolveScheduling} describes that course scheduling at colleges is an optimization problem to be solved under multiple constraints. The most important tasks of course scheduling should consider various constraints, such as conflicts in teaching hours, meeting teacher preferences, and the continuity of teaching hours, etc. These constraints have been made by the University of Twente too and will be checked by our research to see if they really comply. Wang's study utilised genetic algorithm methods to deal with the multiple constraints issue. The results of this study indicated a significant reduction in the amount of time required for course scheduling, and the results were seen as more acceptable by teachers.

Another study conducted in 2003 \cite{designAndImplementationOfACourseSchedulingSystem} researches the design and implementation of a course scheduling system. They are using Tabu Search to accomplish this, which allows them to define strategies and procedures for the timetabling process. They distinguish hard requirements, like no room can have two lectures at the same time, and soft requirements. The hard requirement always have to be met, the soft requirements have weights and determine the quality of the timetable. The used algorithm first makes an initial timetable, and then optimizes the timetable for the soft requirements. After that the room assignment is improved, which at first has a very low priority. This last phase does not change the time scheduling any more. Finally the generated timetables have been compared to the manual solutions of The Business School. It was proven useful for a problem where students could choose any course they want, which manually was being dealt with in not an optimal way.
The used techniques give a good idea about the timetabling problem and help us understand what KPIs (Key Performance Indicators) are good to implement as requirements for timetabling.

Lastly, a rather humorous study was published in 2007 characterizing college students’ daily alcohol consumption patterns and the relation between Thursday drinking and Friday classes overall and for specific vulnerable groups \cite{collegeStudentAlcoholConsumption}. Excessive drinking on Thursday, relative to other weekdays, was found and was moderated by Friday class schedule: hierarchical linear models indicated that students with no Friday classes drank approximately twice as much on Thursdays as students with early Friday classes. Students who had classes beginning at 12 PM. or later consumed similar amounts as those with no Friday classes. The magnitude of the Friday class effect was comparatively larger among males. Ancillary analyses based on the subset of students who showed within-subject variability in Friday classes across semesters (i.e., had both early and late or no Friday classes) produced findings similar to those based on the entire sample. In our case, it can be interesting for the UT to examine these findings since Friday classes, especially those before 10 AM, may reduce excessive drinking.
