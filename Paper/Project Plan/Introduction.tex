\section{Introduction}
Since the start of the very first educational year at the university, schedules have been made to provide a more or less regulated way for providing a solid base where students have the possibility to attend lectures in a decent lecture hall. Although this schedule often contained its faults and flaws, it has been accepted as a tolerable way of organising and assigning the available spaces of the university. Intriguingly, in all those years there has never been executed a thorough analysis on the performance of the schedules. Possibly, many improvements could be made to establish a more beneficiary schedule for every stakeholder or, at least, the most important members of the organisation. The university has come up with a few key performance indicators (KPIs), so it is about to time to actually check whether or not they are actually respected.

To perform an analysis most relevant for the next few years, two datasets containing the schedules for the educational years 2013-2014 and 2014-2015 respectively were analysed. The university’s own KPIs will be checked to see if the schedules actually satisfy their targeted requirements. Additionally, multiple investigations will be executed on interesting and outstanding facts and figures notable in the timetables.