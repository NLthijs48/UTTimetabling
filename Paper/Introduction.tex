\section{Introduction}
With nearly 10.000 students and 3000 staff members, the University of Twente is a very large organisation\cite{stats_uTwente}. Moreover, it is an organisation in which every division has certain privileges and where every single member wants to function as optimal as possible. In achieving this, scheduling plays an enormous role in satisfying the needs of every stakeholder within the organisation. Counting over twenty bachelor studies and over thirty master programmes, not only research facilities, offices, workspaces, communal areas and food courts are of importance to a well-function environment. All these different study programmes need college rooms for lectures, presentations and tuition too.

Since the start of the very first educational year at the university, schedules have been made to provide a more or less regulated way for providing a solid base where students have the possibility to attend lectures in a decent lecture hall. Although this schedule often contained its faults and flaws, it has been accepted as a tolerable way of organising and assigning the available spaces of the university. Intriguingly, in all those years there has never been executed a thorough analysis on the performance of the schedules. Possibly, many improvements could be made to establish a more beneficiary schedule for every stakeholder or, at least, the most important members of the organisation. The university has come up with a few key performance indicators (KPIs), so it is about to time to actually check whether or not they are actually respected.

To perform an analysis most relevant for the next few years, two datasets containing the schedules for the educational years 2013-2014 and 2014-2015 respectively were analysed. The university’s own KPIs were checked to see if the schedules actually satisfy their targeted requirements. Additionally, multiple investigations were executed on interesting and outstanding facts and figures notable in the timetables.

This paper describes the evident events obtained by in-depth analysing the datasets of schedules from the past two educational years, finding that actually KPIs are not really matched at all and that by creating better algorithms and defining better KPIs, there is a lot to 
gain considering optimisation.