\documentclass{sig-alternate-br} % Use the layout as requested by the university

%===== Used packages =====
\usepackage[utf8]{inputenc}	% Use UTF8 characters
\usepackage{url}			% Proper url in the references section
\usepackage{flushend}		% Balance the columns on the last page
\usepackage{relsize}		% Provide \mathlarger to get formula's to the correct size
\usepackage{enumitem}		% Numbered items
\usepackage{placeins}		% Used for \FloatBarrier
\usepackage{float}
\usepackage{verbatim}		% Used for \begin{comment} \end{comment}
\setlist{itemsep=0pt, topsep=2pt}
\usepackage{hyperref}		% Internal references can be clicked
\pagenumbering{arabic} 	% Page numbering
\usepackage{booktabs} % better tables
% Setup captions styling
\usepackage{caption}
\captionsetup[lstlisting]{format=plain, singlelinecheck=false, margin=0pt, font={bf,footnotesize}, justification=centering}
\captionsetup[figure]{format=plain, singlelinecheck=false, margin=0pt, font={bf,footnotesize}, justification=centering, aboveskip=3pt}

%===== Code snippet styling =====
\newcommand{\code}[1]{\texttt{\small \color{inline}#1}} % \code command for inline snippets
\usepackage{listings}		% Code snippets
\usepackage{color}			% Code highlighting colors
\definecolor{dkgreen}{rgb}{0,0.6,0}
\definecolor{inline}{rgb}{0,0,0.5}
\definecolor{gray}{gray}{0.2}
\definecolor{codeText}{rgb}{0.1,0.1,0.1}
\definecolor{mauve}{rgb}{0.58,0,0.82}
\lstset{frame=tb,
  framerule=0.2pt,
  language=Java,
  aboveskip=-2mm,
  belowskip=0mm,
  showstringspaces=false,
  columns=flexible,
  basicstyle={\small\ttfamily\color{codeText}},
  numbers=left,
  numbersep=4pt,
  numberstyle=\tiny\color{gray},
  keywordstyle=\color{blue},
  commentstyle=\color{dkgreen},
  stringstyle=\color{mauve},
  breaklines=true,
  breakatwhitespace=true,
  tabsize=4,
  morekeywords={data, day, dateGiven, daygiven, activity, startTime, endTime, studentsets, room, coursename, teacher}
}
\lstset{language=C, escapechar=$} % Set default language to C
% Set titles for listing references
%\renewcommand\lstlistingname{Algorithm}
%\renewcommand\lstlistlistingname{Algorithms}
%\def\lstlistingautorefname{Algorithm}
% Enable using math mode in code
\lstset{
  mathescape,         
  literate={->}{$\rightarrow$}{2}
           {ε}{$\varepsilon$}{1}
}

\begin{document}

\title{Data Science: Timetabling University of Twente}

\numberofauthors{2}
\author{
\alignauthor Thijs Wiefferink\\
       \affaddr{University of Twente}\\
       \affaddr{P.O. Box 217, 7500AE Enschede}\\
       \affaddr{The Netherlands}\\
       \email{t.w.wiefferink@student.utwente.nl}
\alignauthor Patrick van Looy\\
       \affaddr{University of Twente}\\
       \affaddr{P.O. Box 217, 7500AE Enschede}\\
       \affaddr{The Netherlands}\\
       \email{p.vanlooy@student.utwente.nl}
}
\date{\today}


\maketitle

\begin{abstract}
The aim of this paper is to investigate the timetabling data of the University of Twente and give a performance indication of how well-scheduled the timetables are. Furthermore, the key performance indicators (KPIs) of the University itself are matched with the availbe data to check their compliance. The results were obtained by translating the available spreadsheet data to SQL create and insert statements such that a database could be set up. Next, the relevant data was transformed into XML in order to use XQuery for obtaining relevant results. The results have shown that the KPIs of the university itself were not met in many cases. Plus, interesting trends were obtained by trying out different queries.
\end{abstract}

\keywords{Data Science, Timetable, Scheduling, XML, Databases, XQuery, XPath.}

\section{Introduction}
Since the start of the very first educational year at the university, schedules have been made to provide a more or less regulated way for providing a solid base where students have the possibility to attend lectures in a decent lecture hall. Although this schedule often contained its faults and flaws, it has been accepted as a tolerable way of organising and assigning the available spaces of the university. Intriguingly, in all those years there has never been executed a thorough analysis on the performance of the schedules. Possibly, many improvements could be made to establish a more beneficiary schedule for every stakeholder or, at least, the most important members of the organisation. The university has come up with a few key performance indicators (KPIs), so it is about to time to actually check whether or not they are actually respected.

To perform an analysis most relevant for the next few years, two datasets containing the schedules for the educational years 2013-2014 and 2014-2015 respectively were analysed. The university’s own KPIs will be checked to see if the schedules actually satisfy their targeted requirements. Additionally, multiple investigations will be executed on interesting and outstanding facts and figures notable in the timetables.
\section{Related Work}
The research of Burke and Petrovic \cite{recentResearchDirectionsInAutomatedTimetabling} introduce some approaches to timetabling problems that were developed or are under development at that time in the Automated Scheduling, Optimisation and Planning Research Group
(ASAP) at the University of Nottingham. Their aim was basically the same as in our research since they specifically concentrated upon analysing and improving university timetabling as well. The paper suggests a number of approaches and comprises three parts. Firstly, recent
heuristic and evolutionary timetabling algorithms are discussed. In particular, two evolutionary algorithm developments
are described: a method for decomposing large real-world timetabling problems and a method for heuristic
initialisation of the population. Secondly, an approach that considers timetabling problems as multicriteria decision
problems is presented. Thirdly, it discusses a case-based reasoning approach that employs previous experience to solve
new timetabling problems which basically is similar to our research where we will be analysing past schedules to give indications for improving future schedules.

Paper \cite{designAndImplementationOfACourseSchedulingSystem} is about the design and implementation of a course scheduling system. They are using Tabu Search to accomplish this, this allows them to define strategies and procedures for the timetabling process. 
akes an initial timetable, and then optimizes the timetable for the soft requirements. After that the room assignment is improved, which at first has a very low priority. This last phase does not change the time scheduling any more.

\section{Materials and Methods}
\subsection{Given data}
For this project data about the time tables of the University of Twente is used, this data is provided by the University itself. The data is stored across a couple of \code{.xlsx} files, which can be used with \emph{Microsoft Excel}. There are separate files for the year 2013-2014 and the year 2014-2015. The following files have been provided:

\begin{itemize}
	\item \textbf{Activities:} Rows with the course name, lecture type, date (day, start/end time), teacher, group size, student sets, room
	\item \textbf{Course codes:} Rows with the activity name, description, course code, lecture type, date (day, start/end time) and group size
	\item \textbf{Teachers:} Rows with course code, course name, teacher code and teacher name
	\item \textbf{Usage counts:} Per activity of a certain day of the year a count of the number of people actually in the room.
	\item\textbf{Rooms:} Information about the available equipment in certain rooms
\end{itemize}

The activities and course codes files are actually used for the research in an automated way, the other files are only used to provide context.

\subsection{Excel to SQL}
To work with the provided data it has to be converted to a format that is easily to loop through or query in. The first step of the conversion is to get the data in a SQL database. The Excel file has been saved as tab-separated file through Excel itself. Then a Java program has been written to transform the tab-separated file into a SQL file that has insert statements for importing it into a database. This program first prints a \code{CREATE TABLE} statement to the output, which creates the table in the database with the correct columns. After this the program loops through the lines in the tab-separated file, performs a couple regular expressions on the line and adds it as an \code{INSERT} statement to the output.

The activities data conversion to SQL has just a couple steps, first filter forbidden characters like \code{'} and \code{"}, then replace all tabs by comma-space and as list add the start and end of the \code{INSERT} statement.

The courses data required more work, the main reason for this was that the data has a course code and a module code column. Normally only one of these should be filled in, the course code if it is an old course, the module code if it is a course in the new TOM model. This was however not true in practise, some activities had both and others had neither. Therefore extra corrections on the data have been made with regular expressions in addition to the corrections also made for the activities.

After the script converted the activities and courses data for both years to SQL scripts these have been imported in a PostgreSQL database. This database has been chosen because it is capable of generating XML with SQL queries.

\subsection{SQL to XML} \label{subsec:sql2xml}
From the SQL database the data has been exported as a couple XML databases. For both years a database with the activities and a database with the courses has been exported. The SQL query for generating the activities XML database can be found in Listing \ref{lst:sql2xml}. This database has been used for further querying with XQuery, which is described in Section \ref{subsec:xquery}. In order to get a valid XML database a root element had to be added to the output of the SQL query.

\begin{lstlisting}[caption=SQL to XML conversion, label=lst:sql2xml, float=htpb, language=sql]
select
	xmlelement(name "day",
		xmlforest(days.dateGiven, days.daygiven),
		(select
			xmlagg(xmlelement(name "activity",
				xmlforest(t.starttime, t.endtime, t.studentsets, t.room, t.coursename, t.teacher)
			))
		from
			ut1314 as t
		where
			days.dateGiven = t.dateGiven
		)
	)
from
	(select distinct
		t.dateGiven, t.daygiven
	from
		ut1314 as t
	order by
		t.dateGiven
) as days;
\end{lstlisting}

As shown in the query of Listing \ref{lst:sql2xml}, the data has been organized by day. So this means the XML database contains an element for each day, and this day element contains \code{<activity>} elements. Activity elements contain the course name/code, teacher, location, lecture type, etcetera. This organisation per day helped with the questions that would need to be answered. We for example investigated the number of wasted hours per day, which is defined as hours between lectures. So in this case having all activities of a day together is convenient. Some other questions did not rely on the day grouping, but also did not suffer because of this decision. More about this is explained in Section \ref{subsec:xquery}. The structure of the generated XML database can be found in Listing \ref{lst:xmlStructure}

\begin{lstlisting}[caption=XML structure, label=lst:xmlStructure, float=htpb]
<data>
	<day>
		<dateGiven>2013-08-23</dateGiven>
		<daygiven>Friday</daygiven>
		<activity>
			<startTime>08:45:00</startTime>
			<endTime>10:30:00</endTime>
			<studentsets>IDE M 1A A;CEM-CME M 1A A</studentsets>
			<room>HB 2A</room>
			<coursename>TW M2 Lineaire OptimalisatieZGB/06/01</coursename>
			<teacher>T.W. Wiefferink</teacher>
		</activity>
		...
	</day>
	...
</data>
\end{lstlisting}

\subsection{XQuery: gathering statistics} \label{subsec:xquery}
From the initial databases generated by the SQL queries, as discussed in Section \ref{subsec:sql2xml}, a couple of databases have been generated to specifically check the KPI's. The list below shows details about the generated databases. 

The source databases are generated by SQL, then there is a XQuery script to transform from these database to the top level items of the list. Then there is also a XQuery script for the transformation of each item to its children. In total there are 14 XQuery scripts for these transformations.

After generating the specific databases these are used to get the data for the KPI's. For this there is a XQuery script that goes through these databases and counts the number required for the KPI. The scripts itself can be found on GitHub\footnote{https://github.com/NLthijs48/UTTimetabling}, the data itself cannot be found there because the University of Twente asked to no publish it.

\textbf{Activities database:}
\begin{enumerate}
	\item Activities per day, student sets separated as different activities.
	\begin{enumerate}
		\item Per day for each student set the contactminutes, collegeminutes and start/end time.
		\begin{enumerate}
			\item Count of days that student sets have evening classes and next day early classes.
		\end{enumerate}
	\end{enumerate}
	\item Activities per day, student sets separated, classes extended by 15 minutes
	\begin{enumerate}
		\item Per day for each student the contactminutes, collegeminutes and start/end time.
		\begin{enumerate}
			\item 'Wasted' minutes per student set.
		\end{enumerate}
	\end{enumerate}
	\item Activities per day, teachers separated.
	\begin{enumerate}
		\item Per day for each student set the contactminutes, collegeminutes and start/end time.
		\begin{enumerate}
			\item Count of days that student sets have evening classes and next day early classes.
		\end{enumerate}
	\end{enumerate}
	\item Activities per day, teachers separated, classes extended by 15 minutes.
	\begin{enumerate}
		\item Per day for each teacher the contactminutes, collegeminutes and start/end time.
		\begin{enumerate}
			\item 'Wasted' minutes per teacher.
		\end{enumerate}
	\end{enumerate}
	\item Per room the number of minutes it has been used.
\end{enumerate}

\textbf{Courses database:}
\begin{enumerate}
	\item Per quartile per course the number of planned minutes.
\end{enumerate}

\subsection{Per Quartile statistics and display}
To expand on the KPI statistics there has been an exploration of the data based on per quartile statistics. To generate statistics per quartile instead of per year (as with the KPI's) a Java program has been written. The goal is to show these statistics on a website with bar graphs per quartile. To generate files that can be displayed on a website Java is a good choice, since it can do a lot more than XQuery can. Since Java would already be used for generating the data for display on a website, and the fact that grouping results per quartile in XQuery is complicated, we decided to generate these statistics with Java instead.

As input for the Java program the XML databases as described in Section \ref{subsec:xquery} have been used. 






\section{Results}
After having translated the datasets into more workable and query-able formats, which on its own is quite a result already, the KPIs could be checked for their compliance. By writing and querying rather complex XQueries, we were able to check the compliance of the KPIs. See TABLE ??? for the obtained results.

TODO: INSERT TEXT EXPLAINING FIGURES/TABLES

Not only did we check the KPIs given by the University itself but we also came up with a few other analyses to point out some interesting discoveries in the data sets. As mentioned before, these results are displayed on the website \url{http://wiefferink.me/TimeTabling}.
\section{Discussion}
To mention:
\begin{itemize}
	\item Too much data conversion, rather do it once
	\item automating XQuery is hard, but might have saved time	
	\item students assigned to multiple sets, changes statistics
	\item splitting on student sets and teachers changes statistics
	\item teachers are sometimes split by space, we only detect \code{;}
	\item wasted hours: breaks included
\end{itemize}
\section{Conclusion}
To conclude, we have seen that the KPIs set as a target by the University itself are not that well defined and certainly not achieved in most cases. Elaborating on this, the results show that the University at least tries some form of implementing the KPIs in the timetable algorithms. However, we suspect that there are many improvements possible to boost the compliance rates of the KPI with correct feedback about the final schedules in the form of compliance rates.

Moreover, with more data and specifically more accurate and consistent data, it becomes possible to do a more in-depth analysis on different aspects. For example, it could be interesting to see how the different locations for a particular student are spread out around the campus. With the current data set, this is nearly impossible but when more data is available (possibly a dataset with distances between different locations) this becomes more doable.
\section{Appendices}
\subsection{KPI and Exploration results}
\begin{table*}[]
	\centering
	\caption{KPI and Exploration results}
	\label{table:kpiAndExploration}
	\begin{tabular}{lll}
		\hline
		\textbf{KPI}                                                                                                                                                                                                                                       & \textbf{2013-2014}                                                      & \textbf{2014-2015}                                                    \\ \hline
		Students have a minimum of 4 contact hours on any day                                                                                                                                                                                              & 75.06\%                                                                 & 71.04\%                                                               \\
		Students have a maximum of 6 contact hours on any day                                                                                                                                                                                              & 58.84\%                                                                 & 63.07\%                                                               \\
		Students have a maximum of 2 free hours in 1 series on any day                                                                                                                                                                                     & 81.29\%                                                                 & 70.93\%                                                               \\
		\begin{tabular}[c]{@{}l@{}}The timetable of students have a maximum of 11 college hours on any\\ day. This means 8:15 clock hours, which is the time between start of\\ the first college and the end of the last college on any day)\end{tabular} & 71.69\%                                                                 & 70.44\%                                                               \\
		If a student has a class at the 11th and 12th college hour, \\then that student has no class at the 1st and 2nd college hour the next day                                                                                                            & Broken 269 times                                                        & Broken 99 times                                                       \\
		At Fridays there are no evening classes                                                                                                                                                                                                            & Broken 232 times                                                        & Broken 255 times                                                      \\
		A teacher has a maximum of 8 contact hours per day                                                                                                                                                                                                 & 94.39\%                                                                 & 93.64\%                                                               \\
		\begin{tabular}[c]{@{}l@{}}If a teacher has a class at the 11th and 12th college hour, then that\\ teacher has no class at the 1st and 2nd college hour the next day\end{tabular}                                                                  & Broken 23 times                                                         & Broken 35 times                                                       \\
		\begin{tabular}[c]{@{}l@{}}Rooms must have an occupation of at least 70\%. Occupation is defined\\ as follows: occupying a space (room) by the timetabling process during\\ educational weeks\end{tabular}                                         & 0.37\% (just one room)                                                  & 0.00\%                                                                \\
		Student set with the most 'wasted' time                                                                                                                                                                                                            & \begin{tabular}[c]{@{}l@{}}ATLAS B1 SEM1 A\\ 10320 minutes\end{tabular} & \begin{tabular}[c]{@{}l@{}}ITC AES-ERE 01\\ 8850 minutes\end{tabular} \\
		Teacher with the most 'wasted' time                                                                                                                                                                                                                & \begin{tabular}[c]{@{}l@{}}WW Wits\\ 10320 minutes\end{tabular}         & \begin{tabular}[c]{@{}l@{}}G Meinsma\\ 7860 minutes\end{tabular}     
	\end{tabular}
\end{table*}
\section{Acknowledgements}
We would like to use this opportunity to thank the University of Twente for kindly sharing the datasets of the past two educational years concerning the timetabling schedules with us. Special thanks to Djoerd Hiemstra and Rudy Oude Vrielink for introducing the topic of XML and timetabling.

\bibliographystyle{ieeetr}
\bibliography{TimeTabling}

\end{document}
