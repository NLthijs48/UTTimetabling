\section{Discussion}
To mention:
\begin{itemize}
	\item Too much data conversion, rather do it once
	\item automating XQuery is hard, but might have saved time	
	\item students assigned to multiple sets, changes statistics
	\item splitting on student sets and teachers changes statistics
	\item teachers are sometimes split by space, we only detect \code{;}
	\item wasted hours: breaks included
\end{itemize}

Although results seem quite accurate with such large and inconsistent datasets, results are always skewed. This is certainly not different in our case. The datasets show many inconsistencies which already made the first step of translating the Excel data into an insert query for the database a very complex job. Firstly, many filters needed to be applied to obtain a working insert query. It is possible that some of these filters have altered data in a few cases, however this is quite impossible to validate.

\subsection{Assumptions}
We were given a set of KPIs and questions to perform analyses with on our datasets. However, it was not obvious in all cases how to interpret these. Therefore, we had to make certain assumptions:
\begin{itemize}
	\item We defined a contact hour or college hour as 45 minutes (rather than a clock hour)
	\item We assumed that student sets and teacher were separated with a semicolon
	\item Evening classes are classes after the 9th (college) hour (after 17:30 or 5.30 PM)
	\item Interpretation of KPI 4 in table
	\item\textbf{Rooms:} Information about the available equipment in certain rooms
\end{itemize}

\subsection{Corrections}
We were given a set of KPIs and questions to perform analyses with on our datasets. However, it was not obvious in all cases how to interpret these. Therefore, we had to make certain assumptions:
\begin{itemize}
	\item We defined a contact hour or college hour as 45 minutes (rather than a clock hour)
	\item We assumed that student sets and teacher were separated with a semicolon
	\item 
	\item \textbf{Usage counts:} Per activity of a certain day of the year a count of the number of people actually in the room.
	\item\textbf{Rooms:} Information about the available equipment in certain rooms
\end{itemize}